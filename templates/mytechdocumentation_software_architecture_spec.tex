\documentclass[11pt,letterpaper]{report}
\usepackage{graphicx}

%\usepackage[latin1]{inputenc}
\usepackage[T1]{fontenc}
\usepackage{amsmath}
\usepackage{amssymb}

\usepackage{fixltx2e}

\usepackage{longtable}
%\usepackage{float}
\usepackage{wrapfig}
\usepackage{soul}
\usepackage{textcomp}
\usepackage{marvosym}
\usepackage{wasysym}
\usepackage{latexsym}
\usepackage{geometry}
%\usepackage{subfig}
\usepackage{subcaption}
\usepackage{tabularx}
\usepackage{fullpage}

\usepackage{floatrow}
\usepackage{hyperref}
\hypersetup{
    colorlinks,
    citecolor=blue,
    filecolor=blue,
    linkcolor=blue,
    urlcolor=blue
}
\usepackage{listings}
\usepackage{xcolor}


\usepackage{multirow}
%\usepackage{fontspec}
%\setmainfont{FangSong}
\usepackage{fullpage}
\usepackage{array}
\newcommand{\PreserveBackslash}[1]{\let\temp=\\#1\let\\=\temp}
\newcolumntype{C}[1]{>{\PreserveBackslash\centering}p{#1}}
\newcolumntype{R}[1]{>{\PreserveBackslash\raggedleft}p{#1}}
\newcolumntype{L}[1]{>{\PreserveBackslash\raggedright}p{#1}}

%

%\lstset { %
%     language=C,
%     backgroundcolor=\color{black!5}, % set backgroundcolor
%     basicstyle=\footnotesize,% basic font setting
%                    keywordstyle=\color{blue}\ttfamily,
%                 stringstyle=\color{red}\ttfamily,
%                 commentstyle=\color{green}\ttfamily,
%                 morecomment=[l][\color{magenta}]{\#}
%               }

              \definecolor{mygreen}{rgb}{0,0.6,0}
\definecolor{mygray}{rgb}{0.5,0.5,0.5}
\definecolor{mymauve}{rgb}{0.58,0,0.82}

\lstset{ %
  backgroundcolor=\color{white},   % choose the background color
  basicstyle=\footnotesize\scriptsize,        % size of fonts used for the code
  breaklines=true,                 % automatic line breaking only at whitespace
  captionpos=b,                    % sets the caption-position to bottom
  commentstyle=\color{mygreen},    % comment style
  escapeinside={\%*}{*)},          % if you want to add LaTeX within your code
  keywordstyle=\color{blue},       % keyword style
  stringstyle=\color{mymauve},     % string literal style
}

\begin{document}


\newtheorem{lem}{\bf Lemma}%[section]
\newtheorem{rem}{\bf Remark}%[section]
\newtheorem{pro}{\bf Proposition}%[section]
\newtheorem{defn}{\bf Definition}%[section]
\newtheorem{thm}{\bf Theorem}%[section]
\newtheorem{assu}{Assumption}%[section]
\newcommand{\tr}{{\rm tr}}
\newcommand{\ts}{& \hspace{-0.05in}}
\newcommand{\nn}{\nonumber}
\newtheorem{ex}{Example}[section]
\newcommand{\bp}{\bigskip}
\newcommand{\slp}{\smallskip}
\newcommand{\diag}{{\rm diag}}
\newcommand{\sign}{{\rm sign}}

\newcommand{\rank}{{\rm rank}}
\newcommand{\qed}{\hfill \ensuremath{\Box}}
%\newcolumntype{L}[1]{>{\raggedright\arraybackslash}p{#1}}

\newfloatcommand{capbtabbox}{table}[][\FBwidth]



\title{\bf \LARGE XXX System Software Architecture Specification}

\author{Yu Liu}

\date{}
\maketitle


\chapter*{Revision Record}
\begin{table}[ht]
  \centering
  \begin{tabular}{|p{3cm}| p{3cm}| p{3cm}| p{7cm}|}
    \hline
    {\bf Revision}& {\bf Date}& {\bf Author}& {\bf Content} \\
    \hline
    V1      & 9/24/2018 & Yu Liu & Lateral controller design at
                                   Conception Ready on September 20,
                                   2018\\
    \hline
                                
  \end{tabular}
\end{table}


\newpage
\tableofcontents
\newpage


\chapter{Introduction}
\section{Purpose}
\section{Scope}
\section{Expected Readers and Reading Suggestions}
\section{Terms and Acronyms}
\section{References}

\chapter{Production Introduction}
\section{Function Introduction}
\section{Operating Environment and Context}
\section{Condition and Constraints}
\subsection{Hardware Constraints}
\subsection{Operating System}
\section{Design Principal and Policy}
\subsection{Design Policy}
\subsection{Coding Specification}
\subsection{Other}

\chapter{Static Design}
\section{Software Flowchart}
\section{Software Architecture}
\section{List of Components}
\section{Component Deployment}

\chapter{Dynamic Design}
\section{Software Mode and Status}
\section{Sequence Diagram}
\section{Scheduler Design}
\section{Timing Design}

\chapter{Software Component}
\section{Component XXX}
\subsection{Component Function Description}
\subsection{Component Interface}

\section{Component YYY}
\subsection{Component Function Description}
\subsection{Component Interface}

\chapter{Resources}
\chapter{Interface Design}
\chapter{Application Parameters and Calibration}
\chapter{Non-Functional Requirements}
\section{Scalability}
\section{Performance}
\section{Safety}
\section{Security}

\begin{thebibliography}{199}
%\setlength{\itemsep}{-2mm}



\bibitem{Miller_eCVT} J. M. Miller, ``Hybrid electric vehicle propulsion system
  architectures of the e-cvt type,'' {\it IEEE Transactions on Power
    Electronics}, vol. 21, no. 3, pp. 756--767, May 2006.

\bibitem{Kuang_HEV101_08} M. Kuang, ``Power split system HEV 101,'' Ford
  Internal Presentation, 2008.


\bibitem{Liang_etal_ELDModelingControl_12} W. Liang, X. Wang, T.
  Chrostowski, R. McGee, J. Doering, and M. Kuang, ``Modeling and control of power
  split hybrid vehicle transaxle system using enhanced lever diagram,''
  Ford Technical Report, SRR-2012-0155, 2012.

\bibitem{LiangEtAlEngSpdCtrl12} W. Liang, C. Okubo, T. Chrostowski, and
  M. Kuang, ``Engine speed control of a power split HEV,'' Ford
  Technical Report, SRR-2012-0156, 2012.

\bibitem{MeyerAPSTCR2Presentation} J. Meyer, et al., ``Advanced Power
  Split Transaxle Controls R2 Review,'' APSTC project TDR presentation. 


\bibitem{MeyerAPSTCR3Presentation} W. Liang, et al., ``Advanced Power
  Split Transaxle Controls R3 Review,'' APSTC project TDR presentation. 

\bibitem{AlcantarCrankingComp15} J. V. Alcantar, R. Johri, and M. Kuang,
  ``Minimizing Vibration of the Power Split Drivetrain During Engine
  Starts and Stops via Cranking Torque Estimation and Compensation,''
  Ford Technical Report, SRR-2015ยกV0151.


\end{thebibliography}
\end{document}
\chapter{ Introduction}
This documentation is to detail the design of the Lateral Controller
Module for the Traffic Jam Pilot (TJP) project. The primary purpose of
this system is to provide steering control requests to the vehicle
steering system in order to achieve desired autonomous driving
maneuvers. This system obtains the vehicle's current status in the
lateral direction and its respective path planned to tread. The output
is converted to vehicle interface scaling in another subsystem for
direct steering control of vehicle.

\section{Purpose}
The purpose of this document is to provide a detailed overview of the
Lateral Control Library. The audience for this documentation are
engineers who are intending on creating steering control targets to
control vehicle's headway and steering state. This document should
provide an understanding of the benefits and also detail some of the
heuristic nuances and current problems with the controller and overall
strategy.

\section{Scope}
\label{sec:scope}

\begin{enumerate}
\item The software system created by this system shall be referred to as
  the Lateral Control Module.
\item This software system shall perform feed forward and closed loop
  vehicle control.
\item The primary objective of this system is to take the trajectory,
  vehicle state and the feedback from the brake and steering sensors and
  then determine the desired steering needed to move the vehicle as per
  the next waypoint. The output from this controller is then sent to the
  user interface for implementing the heading of the vehicle.
\end{enumerate}
\section{Definitions, Acronyms, and Abbreviations}
\begin{table}[!htbp]
  \begin{center}
    \caption{Definitions, Acronyms, and Abbreviations.}
    \label{tab:acronyms}
    \begin{tabular}{|p{3cm}|p{10cm}|}
      \hline
      {\bf Abbreviations} & {\bf Description} \\
      \hline
      ACC & Adaptive Cruise Control\\
      \hline
      CAN & Controller Area Network\\
      \hline
      AEB & Automatic Emergency Braking\\
      \hline
      CIPV & Closest In-Path Vehicle \\
      \hline
      ECM & Engine Control Module \\
      \hline
      HV  & Host Vehicle \\
      \hline
      ISO & International Standards Organization\\
      \hline
      RX  & Receive \\
      \hline
      SAS & Steering Angle Sensor \\
      \hline
      TBD & To Be Determined \\
      \hline
      TJP & Traffic Jam Pilot \\
      \hline
      TX  & Transmit \\
      \hline
      
    \end{tabular}
    \end{center}
\end{table}

\chapter{System Design}

\section{System Overview}
\begin{figure}[htbp]
  \centerline{\includegraphics[width=6in]{Lateral_Control_Architecture.jpg}}
  \caption{Lateral Control Architecture.}
  \label{lateral_control_architecture}
\end{figure}
The system architecture of the lateral control system is shown in Figure
\ref{lateral_control_architecture}.

This lateral controller determines the targeted steering depending on
the path to be followed. For this controller geometric path tracking
using a non-linear functions for the lateral distance error is used.
This lateral error distance is measured from the center of the front
wheel to the nearest path point and the heading error with respect to
the target point. The main subsystems for this lateral control can be
described as follows.

% \subsection{Target and Error Determination}
% The {\it Target and Error Determination} module calculates target and
% error variables needed for lateral control calculation. This module
% consists of two subsystems, Vehicle State Generator and Yaw Rate Target
% Subsystem, and their functions are described briefly below.

% \subsubsection{Vehicle State Generator}
% Vehicle State Generator calculates a set of look-ahead values,
% namely look-ahead distance, look-ahead lateral offset, look-ahead
% heading offset based on the current desirable path information. The
% look-ahead values describe the path characteristics at the look-ahead
% distance and will support the control signal calculation for a smoother
% steering commanding. The inputs and outputs of the module of Vehicle
% State Generator are summarized in Table \ref{tab:vehicle_state_generator}.
% \begin{table}[hpbt]
%   \caption{Inputs and Outputs of Vehicle State Generator}
%   \label{tab:vehicle_state_generator}
%   \begin{tabular}{|p{4.5cm}|p{1.5cm}|p{1.5cm}|p{7cm}|}
%     \hline
%     {\bf Signal} & {\bf I/O} & {\bf Unit} & {\bf Description} \\
%     \hline
%     LateralCtrlActive           & Input  & Boolean   & Flag for activating lateral
%                                                        control calculation \\
%     \hline
%     HostVehclSPD\_mps           & Input  & m/s       & Host vehicle speed. \\
%     \hline 
%     LateralOffset\_Current      & Input  & m         & Coefficient $a_0$
%                                                        of the path at
%                                                        vehicle location
%     \\
%     \hline
%     HeadingOffset\_Current      & Input  & rad       & Coefficient $a_1$
%                                                        of the path at vehicle\\
%     \hline
%     Curvature\_Current          & Input  & 1/m       & Coefficient $a_2$
%                                                        of the path at
%                                                        vehicle \\
%     \hline
%     CurvRate\_Current           & Input  & 1/(m$\cdot$ s) & Coefficient
%                                                             $a_3$ of the
%                                                             path at
%                                                             vehicle \\
%     \hline
%     LateralOffset\_LookAhead    &Output   & m         & Coefficient
%                                                         $a_0$ of the
%                                                         path at the
%                                                         look-ahead
%                                                         distance\\
%     \hline
%     HeadingOffset\_LookAhead     & Output  & rad       & Coefficent
%                                                          $a_1$ of the
%                                                          path at the
%                                                          look-ahead
%                                                          distance\\
%     \hline
%     LookAheadDistance            & Output   & m         &Look-ahead
%                                                           distance\\
%     \hline
                                                     
%   \end{tabular}
% \end{table}


% \subsubsection{Yaw Rate Target Sub}
% The module of {\it Yaw Rate Target Sub} calculates a target vehicle yaw
% rate based on the current curvature of the path and vehicle speed. The
% target yaw rate is also used for calculating a yaw rate error. The
% calculation is filtered to reduce the noise level resultant from yaw
% sensor measurement. The inputs and outputs of Yaw Rate Target Sub are
% summarized in Table \ref{tab:yaw_rate_target_sub}.

% \begin{table}[hpbt]
%   \caption{Inputs and Outputs of Yaw Rate Target Sub}
%   \label{tab:yaw_rate_target_sub}
%   \begin{tabular}{|p{4.5cm}|p{1.5cm}|p{1.5cm}|p{7cm}|}
%     \hline
%     {\bf Signal} & {\bf I/O} & {\bf Unit} & {\bf Description} \\
%     \hline
%     Curvature\_Current          & Input  & 1/m       & Coefficient $a_2$
%                                                        of the path at
%                                                        vehicle \\
%     \hline
%     HostVehclSPD\_mps           & Input  & m/s       & Host vehicle
%                                                        speed \\
%     \hline
%     Yaw\_Rate\_Host             & Input  & rad/s     & Host vehicle yaw
%                                                        rate \\
%     \hline
%     Path\_Valid                 & Input  & Boolean   & Flag indicating
%                                                        that the path is
%                                                        valid\\
%     \hline
%     LateralCtrlActive           & Input  & Boolean   & Flag for activating lateral
%                                                        control
%                                                        calculation \\
%     \hline
%     Yaw\_Rate\_Target\_rad      & Output  & rad/s    & Target vehicle
%                                                        yaw rate under
%                                                        current curvature
%                                                        and vehicle
%                                                        velocity\\
%     \hline
%     Yaw\_Rate\_Error\_rad       & Output & rad/s     & Yaw rate error
%                                                        between current
%                                                        vehicle yaw rate
%                                                        and yaw rate
%                                                        target\\
%     \hline
    
%   \end{tabular}
% \end{table}

% \subsection{Steering Feed Forward Control}
% Steering Feed Forward Control determines a feed forward steering command
% signal based on the vehicle speed and the target yaw rate. 

% \begin{table}[hpbt]
%   \caption{Steering Feed Forward Control}
%   \label{tab:feed_forward_control}
%   \begin{tabular}{|p{5.5cm}|p{1.5cm}|p{1.5cm}|p{7cm}|}
%     \hline
%     {\bf Signal} & {\bf I/O} & {\bf Unit} & {\bf Description} \\
%     \hline
%     HostVehclSPD\_mps           & Input  & m/s       & Host vehicle
%                                                        speed \\
%     \hline
%     Yaw\_Rate\_Target\_rad      & Input  & rad/s    & Target vehicle
%                                                       yaw rate under
%                                                       current curvature
%                                                       and vehicle
%                                                       velocity\\
%     \hline
%     SCMD\_YawRate\_FeedForward  & Output & rad     & Feed Forward
%                                                      Steering Command to
%                                                      achieve desired yaw rate at current vehicle speed\\
%     \hline
    
%   \end{tabular}
% \end{table}


% \subsection{Heading Error Feedback Control}

% \subsection{Yaw Rate Error Feedback Control}

% \subsection{Lateral Error Feedback Control}

% \subsection{Steering Command Limits}
% \section{Software Interface}

% \section{Calibrations}



% \begin{figure}[htbp]
% \centerline{\includegraphics[width=5in]{powersplit_diagram}}
% \caption{Model of Power Split HEV Powertrain and Driveline.}
% \label{powersplit_diagram}
% \end{figure}



% \begin{table}
% \begin{center}
% \caption{Characteristics of technical metrics.}
% \label{tab:techmetrics}
% \begin{tabular}{|l|p{11cm}|}
% \hline
% Technical Metrics & Characteristics \\ \hline
% VDV               & Vibration value measured on seat track. Represent
%                     vibration level created via both powertrain mouth
%                     path and half shaft path. \\ \hline
% TDV               & Vibration value measured on half shaft. Represent
%                     vibration level via half shaft path only. \\ \hline
% mVDV              & Vibration value measured in powertrain. Represent
%                     vibration level that is high similar to half shaft
%                     torque vibration. \\ \hline
% \end{tabular}
% \end{center}
% \end{table}


% \begin{figure}[thb]
%     \centering
%     \begin{subfigure}[t]{0.5\textwidth}
%         \centering
%         \includegraphics[width=3in]{EngStartPowerLow}
%         \caption{Smaller power (4 kW). ECT 30$^{\circ}$C, SOC 60\%.}
% \label{fig:EngStartPowerLow}
%     \end{subfigure}%
%     ~ 
%     \begin{subfigure}[t]{0.5\textwidth}
%         \centering
%         \includegraphics[width=3in]{EngStartPowerRegular}
%         \caption{Regular power (6.6 kW). ECT 32$^{\circ}$C, SOC 42\%.}
% \label{fig:EngStartPowerRegular}
%     \end{subfigure}
%     \caption{Impact of start power on engine start smoothness.}
% \label{EngStartPower}
% \end{figure}




% \begin{table}[thpb]
% \caption{Relevant signals in engine stop event.}
% \label{tab:engstopsignals}
% \begin{center}
%   \begin{tabular}{|l|p{5cm}|p{6cm}|}
% \hline
% Signal & Description & Comments
%     \\ \hline
% \verb;OMP_GEN_MODE; &  Generator mode  & Generator mode switches from
%                                          Speed mode (\verb;MG_SPEED;) to
%                                          Engine Stop mode
%                                          (\verb;MG_ENGSTOP;) when engine
%                                          stop events occur. It then
%                                          switches to Torque mode
%                                          (\verb;MG_TORQUE;) when engine
%                                          stop events finish.  
%     \\ \hline
% \verb;PVP_ENG_POS; &  Engine position      & Engine position is
%                                              calculated in the Hybrid
%                                              Transaxle Physical Variable
%     Process (HTPVP).\\ \hline
% \verb;TRP_ESTP_TPOS; & Engine stop target position      & Values come
%                                                           from the
%                                                           calibration
%                                                           array
%                                                           \verb;TRP_ESTP_POS;
%                                                           (Possible
%                                                           engine
%                                                           prepositioning
%     target positions). The final target position is the nearest possible
%     target position to reach, which is calculated from the current
%                                                          engine
%                                                           position,
%                                                           engine speed,
%                                                           and an engine
%                                                           stop speed profile.\\ \hline
% \verb;TRP_ESTP_EXPOS; &  Engine stop expected position    & Expected
%                                                             engine stop
%                                                             position
%                                                             calculated
%                                                             from the
%                                                             engine
%                                                             position at
%                                                             the
%                                                             beginning of
%                                                             the engine
%                                                             stop event
%                                                             as well as
%                                                             the engine
%                                                             stop speed profile. \\ \hline
% \verb;TRP_ESTP_TQ;   &   Engine stop controller command    & PI
%                                                              controller
%                                                              output
%                                                              that is
%                                                              calculated
%                                                              from engine
%     position error between engine position (\verb;PVP_ENG_POS;) and
%                                                              expected
%                                                              engine
%                                                              position (\verb;TRP_ESTP_EXPOS;).\\ \hline
% \verb;TRP_TQ_GEN_CMD; & Arbitrated generator torque command     &
%                                                                   Selected
%     rate-limited generator torque command based on generator mode.\\ \hline
% \verb;TRP_GENTQ_PI;  &    Generator PI controller output   & Final generator
%                                                              torque
%                                                              command
%                                                              from both
%                                                              PI controllers. \\ \hline
%   \end{tabular}
% \end{center}
% \end{table}


% \begin{figure}
% \begin{floatrow}
% \ffigbox{%
% \centerline{\includegraphics[width=3in]{EngStopErrorD0}}
% }
% {

% \caption{Engine stop position errors for Case I.}%
% \label{fig:engstoperrord0}
% }
% \capbtabbox{%
%   \begin{tabular}{|c|c|} \hline
%   Target Position & Error (degree) \\ \hline
%  0                & 30.05 \\ \hline
% 30                & 31.89 \\ \hline
% 60                & 23.11 \\ \hline
% 90                & 18.71 \\ \hline
% 120               & 23.38 \\ \hline
% 150               & 24.94 \\ \hline
% Average           & 25.35 \\ \hline

%   \end{tabular}

% \vspace{1cm}
% }{%
%   \caption{Engine stop position errors and the average error for Case I}%
% \label{table:engstoperrord0}
% }
% \end{floatrow}
% \end{figure}