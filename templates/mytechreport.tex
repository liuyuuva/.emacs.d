\documentclass[11pt,onecolumn,draftclsnofoot,letterpaper]{IEEEtran}
\usepackage{graphicx}

\usepackage[latin1]{inputenc}
\usepackage[T1]{fontenc}
\usepackage{amsmath}
\usepackage{amssymb}

\usepackage{fixltx2e}

\usepackage{longtable}
%\usepackage{float}
\usepackage{wrapfig}
\usepackage{soul}
\usepackage{textcomp}
\usepackage{marvosym}
\usepackage{wasysym}
\usepackage{latexsym}
\usepackage{geometry}
%\usepackage{subfig}
\usepackage{subcaption}
\usepackage{tabularx}

\usepackage{floatrow}

\usepackage{listings}
\usepackage{xcolor}
\usepackage{fullpage}
\usepackage{hyperref}
\hypersetup{
    colorlinks,
    citecolor=blue,
    filecolor=blue,
    linkcolor=blue,
    urlcolor=blue
}
% \lstset { %
%     language=C,
%     backgroundcolor=\color{black!5}, % set backgroundcolor
%     basicstyle=\footnotesize,% basic font setting
%                    keywordstyle=\color{blue}\ttfamily,
%                 stringstyle=\color{red}\ttfamily,
%                 commentstyle=\color{green}\ttfamily,
%                 morecomment=[l][\color{magenta}]{\#}
%               }

              \definecolor{mygreen}{rgb}{0,0.6,0}
\definecolor{mygray}{rgb}{0.5,0.5,0.5}
\definecolor{mymauve}{rgb}{0.58,0,0.82}

\lstset{ %
  backgroundcolor=\color{white},   % choose the background color
  basicstyle=\footnotesize\scriptsize,        % size of fonts used for the code
  breaklines=true,                 % automatic line breaking only at whitespace
  captionpos=b,                    % sets the caption-position to bottom
  commentstyle=\color{mygreen},    % comment style
  escapeinside={\%*}{*)},          % if you want to add LaTeX within your code
  keywordstyle=\color{blue},       % keyword style
  stringstyle=\color{mymauve},     % string literal style
}

\begin{document}


\newtheorem{lem}{\bf Lemma}%[section]
\newtheorem{rem}{\bf Remark}%[section]
\newtheorem{pro}{\bf Proposition}%[section]
\newtheorem{defn}{\bf Definition}%[section]
\newtheorem{thm}{\bf Theorem}%[section]
\newtheorem{assu}{Assumption}%[section]
\newcommand{\tr}{{\rm tr}}
\newcommand{\ts}{& \hspace{-0.05in}}
\newcommand{\nn}{\nonumber}
\newtheorem{ex}{Example}[section]
\newcommand{\bp}{\bigskip}
\newcommand{\slp}{\smallskip}
\newcommand{\diag}{{\rm diag}}
\newcommand{\sign}{{\rm sign}}

\newcommand{\rank}{{\rm rank}}
\newcommand{\qed}{\hfill \ensuremath{\Box}}
\newcolumntype{L}[1]{>{\raggedright\arraybackslash}p{#1}}

\newfloatcommand{capbtabbox}{table}[][\FBwidth]

\title{\bf \LARGE }

\author{Yu Liu}

\date{}
\maketitle


\begin{abstract}


\end{abstract}

\newpage

\tableofcontents

\newpage



\section{\bf Introduction}

\begin{lstlisting}[language=c]
  F32 low_pass_filter(F32 in, F32 w, F32 *last_in, F32 last_out)
{
	F32 out = ((in + *last_in) * (0.5F) * w) + ((1.0F - w) * (last_out));
	*last_in = in;
	return out;
}
\end{lstlisting}


% \begin{figure}[htbp]
% \centerline{\includegraphics[width=5in]{powersplit_diagram}}
% \caption{Model of Power Split HEV Powertrain and Driveline.}
% \label{powersplit_diagram}
% \end{figure}



% \begin{table}
% \begin{center}
% \caption{Characteristics of technical metrics.}
% \label{tab:techmetrics}
% \begin{tabular}{|l|p{11cm}|}
% \hline
% Technical Metrics & Characteristics \\ \hline
% VDV               & Vibration value measured on seat track. Represent
%                     vibration level created via both powertrain mouth
%                     path and half shaft path. \\ \hline
% TDV               & Vibration value measured on half shaft. Represent
%                     vibration level via half shaft path only. \\ \hline
% mVDV              & Vibration value measured in powertrain. Represent
%                     vibration level that is high similar to half shaft
%                     torque vibration. \\ \hline
% \end{tabular}
% \end{center}
% \end{table}


% \begin{figure}[thb]
%     \centering
%     \begin{subfigure}[t]{0.5\textwidth}
%         \centering
%         \includegraphics[width=3in]{EngStartPowerLow}
%         \caption{Smaller power (4 kW). ECT 30$^{\circ}$C, SOC 60\%.}
% \label{fig:EngStartPowerLow}
%     \end{subfigure}%
%     ~ 
%     \begin{subfigure}[t]{0.5\textwidth}
%         \centering
%         \includegraphics[width=3in]{EngStartPowerRegular}
%         \caption{Regular power (6.6 kW). ECT 32$^{\circ}$C, SOC 42\%.}
% \label{fig:EngStartPowerRegular}
%     \end{subfigure}
%     \caption{Impact of start power on engine start smoothness.}
% \label{EngStartPower}
% \end{figure}




% \begin{table}[thpb]
% \caption{Relevant signals in engine stop event.}
% \label{tab:engstopsignals}
% \begin{center}
%   \begin{tabular}{|l|p{5cm}|p{6cm}|}
% \hline
% Signal & Description & Comments
%     \\ \hline
% \verb;OMP_GEN_MODE; &  Generator mode  & Generator mode switches from
%                                          Speed mode (\verb;MG_SPEED;) to
%                                          Engine Stop mode
%                                          (\verb;MG_ENGSTOP;) when engine
%                                          stop events occur. It then
%                                          switches to Torque mode
%                                          (\verb;MG_TORQUE;) when engine
%                                          stop events finish.  
%     \\ \hline
% \verb;PVP_ENG_POS; &  Engine position      & Engine position is
%                                              calculated in the Hybrid
%                                              Transaxle Physical Variable
%     Process (HTPVP).\\ \hline
% \verb;TRP_ESTP_TPOS; & Engine stop target position      & Values come
%                                                           from the
%                                                           calibration
%                                                           array
%                                                           \verb;TRP_ESTP_POS;
%                                                           (Possible
%                                                           engine
%                                                           prepositioning
%     target positions). The final target position is the nearest possible
%     target position to reach, which is calculated from the current
%                                                          engine
%                                                           position,
%                                                           engine speed,
%                                                           and an engine
%                                                           stop speed profile.\\ \hline
% \verb;TRP_ESTP_EXPOS; &  Engine stop expected position    & Expected
%                                                             engine stop
%                                                             position
%                                                             calculated
%                                                             from the
%                                                             engine
%                                                             position at
%                                                             the
%                                                             beginning of
%                                                             the engine
%                                                             stop event
%                                                             as well as
%                                                             the engine
%                                                             stop speed profile. \\ \hline
% \verb;TRP_ESTP_TQ;   &   Engine stop controller command    & PI
%                                                              controller
%                                                              output
%                                                              that is
%                                                              calculated
%                                                              from engine
%     position error between engine position (\verb;PVP_ENG_POS;) and
%                                                              expected
%                                                              engine
%                                                              position (\verb;TRP_ESTP_EXPOS;).\\ \hline
% \verb;TRP_TQ_GEN_CMD; & Arbitrated generator torque command     &
%                                                                   Selected
%     rate-limited generator torque command based on generator mode.\\ \hline
% \verb;TRP_GENTQ_PI;  &    Generator PI controller output   & Final generator
%                                                              torque
%                                                              command
%                                                              from both
%                                                              PI controllers. \\ \hline
%   \end{tabular}
% \end{center}
% \end{table}


% \begin{figure}
% \begin{floatrow}
% \ffigbox{%
% \centerline{\includegraphics[width=3in]{EngStopErrorD0}}
% }
% {

% \caption{Engine stop position errors for Case I.}%
% \label{fig:engstoperrord0}
% }
% \capbtabbox{%
%   \begin{tabular}{|c|c|} \hline
%   Target Position & Error (degree) \\ \hline
%  0                & 30.05 \\ \hline
% 30                & 31.89 \\ \hline
% 60                & 23.11 \\ \hline
% 90                & 18.71 \\ \hline
% 120               & 23.38 \\ \hline
% 150               & 24.94 \\ \hline
% Average           & 25.35 \\ \hline

%   \end{tabular}

% \vspace{1cm}
% }{%
%   \caption{Engine stop position errors and the average error for Case I}%
% \label{table:engstoperrord0}
% }
% \end{floatrow}
% \end{figure}


 

\section{\bf Conclusions}

% \section*{Acknowledgment}
% This research was supported by the National Aeronautics and Space Administration under the grant

%\bibliography{mybib}
%\bibliographystyle{IEEEtran} % Include this if you use bibtex

\begin{thebibliography}{199}
%\setlength{\itemsep}{-2mm}



\bibitem{Miller_eCVT} J. M. Miller, ``Hybrid electric vehicle propulsion system
  architectures of the e-cvt type,'' {\it IEEE Transactions on Power
    Electronics}, vol. 21, no. 3, pp. 756--767, May 2006.

\bibitem{Kuang_HEV101_08} M. Kuang, ``Power split system HEV 101,'' Ford
  Internal Presentation, 2008.


\bibitem{Liang_etal_ELDModelingControl_12} W. Liang, X. Wang, T.
  Chrostowski, R. McGee, J. Doering, and M. Kuang, ``Modeling and control of power
  split hybrid vehicle transaxle system using enhanced lever diagram,''
  Ford Technical Report, SRR-2012-0155, 2012.

\bibitem{LiangEtAlEngSpdCtrl12} W. Liang, C. Okubo, T. Chrostowski, and
  M. Kuang, ``Engine speed control of a power split HEV,'' Ford
  Technical Report, SRR-2012-0156, 2012.

\bibitem{MeyerAPSTCR2Presentation} J. Meyer, et al., ``Advanced Power
  Split Transaxle Controls R2 Review,'' APSTC project TDR presentation. 


\bibitem{MeyerAPSTCR3Presentation} W. Liang, et al., ``Advanced Power
  Split Transaxle Controls R3 Review,'' APSTC project TDR presentation. 

\bibitem{AlcantarCrankingComp15} J. V. Alcantar, R. Johri, and M. Kuang,
  ``Minimizing Vibration of the Power Split Drivetrain During Engine
  Starts and Stops via Cranking Torque Estimation and Compensation,''
  Ford Technical Report, SRR-2015�V0151.


\end{thebibliography}






\end{document}
